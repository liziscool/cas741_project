\documentclass[12pt, titlepage]{article}

\usepackage{booktabs}
\usepackage{tabularx}
\usepackage{hyperref}
\hypersetup{
    colorlinks,
    citecolor=black,
    filecolor=black,
    linkcolor=red,
    urlcolor=blue
}
\usepackage[round]{natbib}

%% Comments

\usepackage{color}

\newif\ifcomments\commentstrue

\ifcomments
\newcommand{\authornote}[3]{\textcolor{#1}{[#3 ---#2]}}
\newcommand{\todo}[1]{\textcolor{red}{[TODO: #1]}}
\else
\newcommand{\authornote}[3]{}
\newcommand{\todo}[1]{}
\fi

\newcommand{\wss}[1]{\authornote{blue}{SS}{#1}} 
\newcommand{\plt}[1]{\authornote{magenta}{TPLT}{#1}} %For explanation of the template
\newcommand{\an}[1]{\authornote{cyan}{Author}{#1}}


%% Common Parts

\newcommand{\progname}{Time\_Freq\_Analysis} % PUT YOUR PROGRAM NAME HERE %Every program
                                % should have a name


\begin{document}

\title{Test Report: \progname{}} 
\author{Elizabeth Hofer}
\date{\today}
	
\maketitle

\pagenumbering{roman}

\section{Revision History}

\begin{tabularx}{\textwidth}{p{3cm}p{2cm}X}
\toprule {\bf Date} & {\bf Version} & {\bf Notes}\\
\midrule
27.12.20 & 1.0 & Initial Release\\
\bottomrule
\end{tabularx}

~\newpage

\section{Symbols, Abbreviations and Acronyms}

\renewcommand{\arraystretch}{1.2}
\begin{tabular}{l l} 
  \toprule		
  \textbf{symbol} & \textbf{description}\\
  \midrule 
  T & Test\\
  STFT & Short Time Fourier Transform\\
  FFT & Fast Fourier Transform\\
  \bottomrule
\end{tabular}\\

\wss{symbols, abbreviations or acronyms -- you can reference the SRS tables if needed}

\newpage

\tableofcontents

\listoftables %if appropriate

\listoffigures %if appropriate

\newpage

\pagenumbering{arabic}

This document review the Verification and Validation specifications as outlined in the VnVPlan (\url{https://github.com/liziscool/cas741_project/blob/master/docs/VnVPlan/VnVPlan.pdf}) for \progname{}. 

\section{Functional Requirements Evaluation}
\begin{itemize}

\item[R1] \emph{Program shall take a sequence of numbers representing the signal to be analysed as input. All other inputs will have defaults, but program shall accept user inputs for those as well.}
\\
Tentatively Met. Due to anomalies/misunderstandings in implementation the frequency range cannot be specified. This feature will either be not implemented (as it may not be necessary), or will be implemented in future updates

\item[R2] \emph{Program shall notify user if an input value is illegal or out of bound}
\\
Met.

  
\item[R3] \emph{The output shall be a time frequency representation of the signal in the specified time period and over the specified frequency range.}\\
Met

\item[R4] \emph{The time-frequency representations of simple input signals (such as sinusoids of a constant frequency or an impulse) should be comparable to existing time-frequency transforms of that signal.}\\
Implimentation of Testing, in progress. 

\end{itemize}
\section{Non-functional Requirements Evaluation}

\begin{itemize}
\item[R5]\emph{ Program shall plot time-frequency representation as a heat map.}\\
Met.
\item[R6] \emph{The time complexity for this program should be $O(n)$.}\\
Honestly I have no idea and I don't even know how I would have gone about testing this. I had previous implementations that took forever and I have since fixed that. It runs 'quick enough' for my purposes. 

\item[R7] \emph{Program will not have a graphical user interface but should still be easy to use, the input parameters besides the signal shall all have default values, there should be at most 6 optional inputs.}\\
Met.


\item[R8] \emph{The program code should be clear and readable.}\\
Partially met. Will be improved in future versions.
\item[R9] \emph{The program should easily integrate with other software programs.}\\
Met. 
\item[R10] \emph{The program should minimize spectral leakage.}\\
Testing of this requirement was not completed. It will be completed in future versions.

\end{itemize}
	
\section{Comparison to Existing Implementation}	

This section will not be appropriate for every project.

\section{Unit Testing}

\section{Changes Due to Testing}

\section{Automated Testing}
		
\section{Trace to Requirements}
		
\section{Trace to Modules}		

\section{Code Coverage Metrics}

\bibliographystyle{plainnat}

\bibliography{../../refs/References}

\end{document}