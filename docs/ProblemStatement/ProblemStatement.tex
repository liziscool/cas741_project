\documentclass{article}

\usepackage{tabularx}
\usepackage{booktabs}

\title{CAS 741: Problem Statement\\Time-Frequency Analysis of Machine Vibration Recordings}

\author{Elizabeth Hofer \\ hofere1}

\date{September 20th, 2020}

\input{../Comments}

\begin{document}

\maketitle

\begin{table}[hp]
\caption{Revision History} \label{TblRevisionHistory}
\begin{tabularx}{\textwidth}{llX}
\toprule
\textbf{Date} & \textbf{Developer(s)} & \textbf{Change}\\
\midrule
9.20.20 & E. Hofer & Initial release of document\\
9.27.20 & E. Hofer & Minor formating updates\\
\bottomrule
\end{tabularx}
\end{table}


Large machines (called \emph{vibrating screens}) are used in the mining and 
aggregate industry to sort gravel by size by exciting the gravel at specific 
frequencies and with specific patterns. For the purpose of this research the 
vibrations of the machines have been recorded and the recordings have been 
obtained. The vibration recordings contain important information when analysed 
correctly, for example one could extract machine identifying characteristics 
(i.\ e.\ a machine fingerprint) or evidence on whether the machine is 
functioning correctly.\\
  
To extract this information from the recording one must analyse the 
time-frequency content of the recording data (i.\ e.\ identify what frequencies 
occur at what time instance of the sample). The trivial approach for finding 
frequency content, Fourier Transforms, only identify what frequencies are 
present in the sample but cannot identify at what time they take place, which 
would be ill suited to the time-specific machine recording data. Therefore, the 
analysis that is taken must relay time-localized frequency information.\\

Essentially, the program should take in a time-domain sample recording and produce a time-frequency domain representation of that sample.\\

Interested stakeholders include researchers who require time-frequency analysis 
of their data, engineers and technicians who work with vibrating machines, and 
more specifically the company providing the data for this research: Haver \& 
Boecker Canada (HBC). The projected is intended to be utilized within a larger 
software framework and as such should be easily integrated with existing code.\\


\end{document}